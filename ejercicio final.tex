\documentclass[10pt,a4paper]{article}
\usepackage[utf8]{inputenc}
\usepackage[spanish]{babel}
\usepackage{amsmath}
\usepackage{amsfonts}
\usepackage{amssymb}
\usepackage{graphicx}
\usepackage{anysize}
\usepackage{wrapfig}
\usepackage{multimedia}
\usepackage{color}
\usepackage[left=2cm,right=2cm,top=2cm,bottom=2cm]{geometry}
\title{Asi funciona el sistema VTEC}
\begin{document}

\begin{Huge}
\begin{center}
\textcolor{red}{\emph{\textbf{Asi funciona el sistema VTEC}}}
\centering
\includegraphics[scale=.60]{vtec2}
\end{center}
\end{Huge}

Si alguna vez te has preguntado qué es el \textbf{sistema VTEC de Honda}, aquí tienes la respuesta. En esta infografía vemos un árbol de levas que va moviendo las válvulas de admisión y escape de un motor de gasolina. Por debajo de 4.500 RPM, la apertura de las mismas es pequeña, para minimizar la relación de aire que se introduce en los cilindros.\\

Cuando se supera ese régimen, el tercer brazo se mueve también, provocando una mayor apertura de las válvulas, \textbf{aireando mejor el cilindro} y mejorando la evacuación de los gases de escape. Con este sistema se consigue un mejor rendimiento a bajas vueltas y que a un régimen elevado el motor responda mejor.\\
\begin{wrapfigure}{l}{80mm}
  \begin{center}
    \includegraphics[scale=.45]{advancedvtec3.jpg}
  \end{center}
  \caption{Honda K20}
\end{wrapfigure}

La fórmula de aceleración que sigue para conseguir este rendimiento es:
\begin{equation}
\delta=\frac{V_f+V_o\cdot t}{2}
\end{equation}

El \textbf{sistema VTEC} (Variable valve Timing and lift Electric Control), fue desarrollado por Honda y en 1989 fue cuando se puso a prueba por primera vez. Básicamente se trata de un sofisticado sistema de distribución variable que ha ido evolucionando pero siempre manteniendo las mismas características principales.\\

El funcionamiento del sistema VTEC es muy famoso a nivel mundial y lo cierto es que hay muchas leyendas urbanas acerca de su funcionamiento. Se trata de una de las características clave en modelos como el Honda Civic que lo hacen destacar de la competencia más directa, pero debemos entender qué es realmente el VTEC y cuál es su función.\\

\pagebreak
Una lista de los motores que usan el sistema VTEC sería:
\begin{enumerate}
\item SOHC:
\begin{enumerate}
\item D15: Honda Civic: 1.5 105hp, 1.5 115hp
\item D16: Honda Civic, CRX del Sol: 1.6 126hp
\end{enumerate}
\item DOHC:
\begin{enumerate}
\item B16: CRX del Sol VTI: 1.6 160hp
\item B18: Civic VTI: 1.8 169hp
\item K20: Civic Type R: 2.0 215hp 
\item K24: Civic Type R: 2.4 245hp
\end{enumerate}
\end{enumerate}

Los últimos modelos, K20 y K24, usan un sistema i-VTEC, que es un sistema idéntico al VTEC tradicional, pero controlando el movimento de levas de forma electrónica.\\

\centering
\includegraphics[scale=.50]{ivtec}
\end{document}